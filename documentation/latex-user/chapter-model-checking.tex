\section{Model checking in \emph{Reprotool}.}

\subsection{Introduction to model checking in \emph{Reprotool}}.

\emph{Reprotool} project can contain many use-cases. Use-cases consis of use-case steps. You can assign annotaions to individual
use-case steps. When the \emph{Reprotool} performs model checking, it firstly initializes all annotaion variables to boolean value false.
It then starts executing use-cases. Executing a use-case means executing its use-case steps. If a step is executed that has
an annotation, the value of that annotation variable is changed to true. The annotaions in a \emph{Reprotool} project must adhere to
some rules that are defined in the project as a set of temporal logic formulas. When you run the model checking in \emph{Reprotool},
you are actually checking the validity of these temporal logic formulas. By assigning new annotaions to use-case steps in the project,
you are actually extending the set of logic formulas that will be verified during the model checking process.

\subsection{Predefined \emph{Reprotool} annotaions an their semantics.}
In every \emph{Reprotool} project, there is already a set of predefined annotations that you can use. You can, however, define new
annotations and define their behaviour. The predefined annotations already have some predefined default behaviour.

\subsubsection {The \emph{open} and \emph{close} annotaions are bound by these rules:}

\begin{enumerate}
  \item After 'open' there should always be 'close'.
  \item No multi-open without close.
  \item No multi-close without open.
  \item First 'open' then 'close'.
\end{enumerate}

\subsubsection{The \emph{create} and \emph{use} annotaions are bound by these rules:}

\begin{enumerate}
  \item After 'create' there must be some branch containing 'use'.
  \item Only one 'create'.
  \item First 'create' then 'use'.
\end{enumerate}

\subsubsection{The special \emph{trace} and \emph{on} annotations}
These two annotations are special, you can not override their meaning. By using these annotations, you can conditionally prune the
execution tree of the model checker during the verification process. You will better understand their meaning when we demonstrate
their usage later in this chapter.